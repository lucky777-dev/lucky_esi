\documentclass{beamer}

\usepackage[francais]{babel}
\usepackage[utf8]{inputenc}
\usepackage[T1]{fontenc}
\usepackage{graphicx}
\usepackage{graphics}
\usepackage{color}
\usepackage{textcomp}
\usepackage{pifont}
\usepackage[normalem]{ulem}
\usepackage{times}
\usepackage{hyperref}
\usepackage{verbatim}
\usepackage{amsmath}
\usepackage{amsthm}
\usepackage{amsfonts}
\usepackage[mathscr]{euscript}
\usepackage{pgfpages}
\usepackage{listings}
\usepackage{subfigure}
\usepackage{algorithm}
\usepackage[noend]{algorithmic}
\usepackage{pdftricks}
\usepackage{mathrsfs}
\usepackage{array}
\usepackage{fancybox}
% \usepackage{columns}
\usepackage{multirow}
\usepackage{url}
\usepackage{tikz}
\usepackage{colortbl}
%\usepackage{cite} %DO NOT FUCKING USE CITE ON BEAMER !!! LOST 30 GODDAM' MINUTES ON THIS SHIT !!!
\usepackage{mathabx}
\usepackage{amssymb}
\usepackage{eurosym}
\usepackage{wasysym} % ch0

\let\texteuro\euro

\hypersetup{colorlinks,%
            citecolor=black,%
            filecolor=black,%
            linkcolor=black,%
            urlcolor=blue}

%\addtolength{\parskip}{10pt}

\usetikzlibrary{calc}

\mode<presentation>
\setbeamertemplate{footline}[frame number]
\setbeamercovered{transparent}
\usetheme[navigation]{ESI}

%lst
\definecolor{comment-green}{RGB}{0,166,80}
\lstset{language=C++,
  keywordstyle=\lst@ifdisplaystyle\bf\fi\color{blue!60},
  commentstyle=\color{comment-green},
  stringstyle=\color{red},
  basicstyle=\lst@ifdisplaystyle\tiny\else\tt\fi,
  morekeywords={
    constexpr,concept,decltype,nullptr,nullptr_t,noexcept,final,override},
  frame=single,
  xleftmargin=0.5cm,
  numbers=left,
  tabsize=2}

%title
\subtitle{Langage \texttt{C} / \cpp}
\author{R. Absil}
\date{\today}

%styles
\theoremstyle{definition}
\newtheorem{thm}{Théorème}
\newtheorem{conj}[thm]{Conjecture}
\newtheorem{deff}[thm]{Définition}
\newtheorem{prop}[thm]{Propriété}
\newtheorem{lem}[thm]{Lemme}
\newtheorem*{lem*}{Lemme}
\newtheorem{cor}[thm]{Corollaire}
%\newtheorem{example}{Exemple}
\newtheorem{remark}{Remarque}
\newtheorem{exo}{Exercice}

%typeset
\newcommand{\ie}{{\emph{i.e., }}}
\newcommand{\eg}{{\emph{e.g., }}}
\newcommand{\etal}{{\emph{et al.}}}
\newcommand{\rrceil}{\unichar{"2308}}
\newcommand{\sloand}[2]{\footnote{N. J. A. Sloane - OEIS Foundation - \texttt{www.oeis.org}, Sequence #1 - #2.}}

%math
\newcommand{\IN}{{\mathbb N}}
\newcommand{\IQ}{{\mathbb Q}}
\newcommand{\IR}{{\mathbb R}}
\newcommand{\IZ}{{\mathbb Z}}
\newcommand{\IP}{{\mathbb P}}
\newcommand{\IC}{{\mathbb C}}
\newcommand{\bigo}{{\mathcal{O}}}
\renewcommand{\mod}{\bmod}
\newcommand{\ssi}{\Leftrightarrow}
\newcommand{\then}{\Rightarrow}
\newcommand{\fle}[1]{\stackrel{#1}{\longrightarrow}}
\newcommand{\suchthat}{~\big|~}
\newcommand{\floor}[1]{\left\lfloor #1 \right\rfloor}
\newcommand{\ceil}[1]{\left\lceil #1 \right\rceil}
\DeclareMathOperator*{\argmin}{argmin}
\DeclareMathOperator*{\argmax}{argmax}

%tikz
\tikzstyle{_vertex}=[fill=white, circle,minimum size=12pt,inner sep=1pt]
\tikzstyle{_blackv}=[fill=black, circle,minimum size=8pt,inner sep=1pt]
\tikzstyle{_dot}=[fill=black, circle, minimum size = 1mm, inner sep=0pt]
\tikzstyle{_bigvertex}=[fill=white, circle,minimum size=21pt,inner sep=1pt]
\tikzstyle{_arc}=[->, >=stealth]
\tikzstyle{_boldarc}=[->, >=stealth, line width=2pt]

\newcommand{\cpp}{\texttt{C++}}
\newcommand{\java}{\texttt{Java}}


\title{Ch. 9 - Exceptions}

\begin{document}
\begin{frame}
  \titlepage
\end{frame}

\begin{frame}
  \frametitle{Table des matières}
  \footnotesize \tableofcontents[pausesections,pausesubsections]
\end{frame}


\section{Introduction}

%intro str str p367

\begin{frame}
\frametitle{Introduction}
\begin{itemize}[<+->]
\item Servent à gérer un comportement \emph{exceptionnel} du programme
	\begin{itemize}
	\item Ne pas abuser du mécanisme.
	\end{itemize}
\item Mécanisme similaire aux exceptions en Java.
	\begin{itemize}
	\item Blocs try/catch.
	\end{itemize}
\item Mécanisme de gestionnaire similaire.
	\begin{itemize}
	\item Ordre des \texttt{catch} a de l'importance.
	\end{itemize}
\item On peut lancer n'importe quel objet en \cpp.
	\begin{itemize}
	\item Pas de superclasse d'exceptions.
	\end{itemize}
\item Plusieurs classes d'exception sont implémentées dans \texttt{stdexcept.h}
\end{itemize}
\end{frame}

\begin{frame}[containsverbatim]
\frametitle{Exemple}
\begin{itemize}
\item Fichier \texttt{array.cpp}
\end{itemize}
\begin{lstlisting}
class array
{
	int n; double * tab;
	
	public:
		double & operator [] (int i)
		{
			rangeCheck(i);
			return tab[i];
		}

	private:
		inline void rangeCheck(int i)
		{
			if(i < 0 || i >= n)
			{
				string s = "out of range : size of ";
				s += to_string(n);
				s += " , accessed at ";
				s += to_string(i);
				throw out_of_range(s);
			}
		}
};
\end{lstlisting}
\end{frame}

\begin{frame}[containsverbatim]
\frametitle{Exemple}
\begin{itemize}
\item Fichier \texttt{array.cpp}
\end{itemize}
\begin{lstlisting}
int main()
{
	array v(5);

	for(int i = 0; i < 5; i++)
		v[i] = i * i;

	for(int i = 0; i < 5; i++)
		cout << v[i] << endl;

	v[0] = 2;
	
	try
	{
		v[-1] = 4;
	}
	catch(const out_of_range& e)
	{
		cout << e.what() << endl;
	}
}
\end{lstlisting}
\end{frame}

\begin{frame}
\frametitle{Garanties d'exceptions}
\begin{itemize}[<+->]
\item Dans la conception d'un programme, on peut fournir diverses garanties en termes d'exceptions
	\begin{enumerate}
	\item \textbf{Aucune exception} : garantie de succès des opérations
	\item \textbf{Garantie forte} : les opérations peuvent échouer, mais sans effet de bord
	\item \textbf{Garantie de base} : les opérations peuvent échouer et elles peuvent avoir des effets de bord, mais les invariants sont préservés et aucune ressource n'est bloquée ou perdue (locks, leaks, etc.)
	\item \textbf{Aucune garantie}
	\end{enumerate}
\item On veut la plus forte garantie possible, mais au minimum « la base »
	\begin{itemize}
	\item Clause \texttt{finally}, \texttt{try with resource}, destructeurs, etc.
	\end{itemize}
\end{itemize}
\end{frame}

\section{Généralités}

\begin{frame}
\frametitle{Spécification de prototype}
\begin{itemize}[<+->]
\item Une fonction peut spécifier les exceptions qu'elle est capable de lancer.
\item Les exceptions non prévues appellent la fonction \texttt{unexpected}
	\begin{itemize}
	\item Peut appeler \texttt{terminate} (non standard, cf. ci-après).
	\item Peut être définie avec \texttt{set\_unexpected}
	\end{itemize}
\end{itemize}
\begin{exampleblock}<+->{Exemple}
	\begin{itemize}
	\item \lstinline|void f() throw (A, B) \{ ... \}|
	\item Similaire à 
		\begin{itemize}
		\item \lstinline|try \{ ... \}|
		\item \lstinline|catch(const A & a) \{ throw; \}|
		\item \lstinline|catch(const B & b) \{ throw; \}|
		\item \lstinline|catch(...) \{ unexpected(); \}|
		\end{itemize}
	\end{itemize}
\end{exampleblock}
\end{frame}

\begin{frame}
\frametitle{Design et performances}
\begin{itemize}[<+->]
\item En \cpp, traiter un bloc try/catch prend autant de temps qu'un bloc normal d'instructions, \emph{si} aucune exception n'est lancée
	\begin{itemize}
	\item Pas de perte de performances
	\end{itemize}
\item Implémenté via des tables statiques
\item À chaque instruction susceptible de lancer une exception, on enregistre dans une table quelques informations permettant de trouver la clause \lstinline|catch| correspondante
	\begin{itemize}
	\item Mécanisme similaire au gestionnaire d'interruptions système
	\end{itemize}
\item Possibilité d'indiquer qu'une fonction ne lance \emph{jamais} d'exception grâce au mot-clé \lstinline|noexcept|
	\begin{itemize}
	\item Offre des optimisations compilatoires
	\end{itemize}
\item Simple indication, pas une contrainte
	\begin{itemize}
	\item Si une exception est lancée au sein d'une fonction \lstinline|noexcept|, \texttt{terminate} est appelée
	\end{itemize}
\end{itemize}
\end{frame}

\begin{frame}[containsverbatim]
\frametitle{Exemple}
\begin{itemize}
\item Fichier \texttt{noexcept.cpp}
\end{itemize}
\begin{lstlisting}
struct A
{
	int f() noexcept { return 1; }
};

struct B : A
{
	int f() { return 2; } //ok, noexcept inherited
};

void f() noexcept;

void f() {} //ko : different exception specifier
\end{lstlisting}
\end{frame}

\section{Contexte de lancement}

\begin{frame}
\frametitle{Portée}
\begin{itemize}[<+->]
\item Une variable automatique crée dans un bloc est locale à un bloc
\item Effectuer un \lstinline|throw| change le contexte
	\begin{itemize}
	\item On change de bloc après un \lstinline|throw|.
	\end{itemize}
\item Destruction synchrone des objets automatiques.
	\begin{itemize}
	\item Pas les objets dynamiques
	\end{itemize}
\end{itemize}
\begin{alertblock}<+->{Problème}
	\begin{itemize}[<+->]
	\item Comment gérer des objets dynamiques dans un bloc \lstinline|try|  ?
	\end{itemize}
\end{alertblock}
\begin{exampleblock}<+->{Solution}
	\begin{itemize}[<+->]
	\item Ne pas créer de tels objets (contraignant).
	\item Utiliser des « pointeurs intelligents ».
	\end{itemize}
\end{exampleblock}
\end{frame}

\begin{frame}[containsverbatim]
\frametitle{Exemple}
\begin{itemize}
\item Fichier \texttt{no-dest.cpp}
\end{itemize}
\begin{lstlisting}
struct A 
{
	~A()
	{
		cout << "-A" << endl;
	}
};

int main()
{
	try
	{
		A * a = new A();
		throw 0;
	}
	catch(const int & i)
	{
		cout << "Error" << endl;
	}
}
\end{lstlisting}
\end{frame}

\begin{frame}
\frametitle{Transmission des exceptions}
\begin{itemize}[<+->]
\item Les exceptions sont toujours transmises par valeur.
	\begin{itemize}
	\item Même si on lance une référence
	\end{itemize}
\item Nécessité : comme on change de contexte, les variables automatiques sont détruites.
\item Si on lance un pointeur, il faut s'assurer qu'il ne pointe pas vers une variable automatique locale au bloc.
\end{itemize}
\begin{exampleblock}<+->{Erreur probable}
	\begin{itemize}[<+->]
	\item \lstinline|A * a = new A;|
	\item \lstinline|throw a;|
	\end{itemize}
\end{exampleblock}
\begin{itemize}[<+->]
\item Bonne pratique : lancer un objet automatique ou un pointeur intelligent. Éviter de lancer du dynamique.
\item En général, on « attrape » les exceptions par référence constante
\end{itemize}
\end{frame}

\begin{frame}[containsverbatim]
\frametitle{Exemple}
\begin{itemize}
\item Fichier \texttt{scope.cpp}
\end{itemize}
\begin{lstlisting}
void f(int& n)
{
	int i = 1;
	try
	{		
		int& j = i;
		n++;
		throw j;
	}
	catch(int& j)
	{
		n++; //n is accessible
		j++;
		cout << "i : " << i << endl; //copied
		cout << "j : " << j << endl;
	}
}
\end{lstlisting}
\end{frame}

\begin{frame}[containsverbatim]
\frametitle{Exemple}
\begin{itemize}
\item Fichier \texttt{scope.cpp}
\end{itemize}
\begin{lstlisting}
void g()
{
	try
	{
		int i = 1;
		int* pti = &i;
		throw pti;
	}
	catch(int* pti)
	{
		cout << "*pti : " << *pti << endl; //undefined behaviour
	}
}
\end{lstlisting}
\end{frame}

\begin{frame}
\frametitle{Gestion des ressources}
\begin{itemize}[<+->]
\item Quand une fonction acquiert une ressource, il est important qu'elle la libère quand elle n'en n'a plus besoin
\item Souvent, la fonction libère la ressource acquise en fin de bloc, avant \lstinline|return|
\end{itemize}
\begin{alertblock}<+->{Problème}
	\begin{itemize}
	\item Que faire si une instruction lance une exception entre l'acquisition et la libération des ressources ?
	\end{itemize}
\end{alertblock}
\begin{itemize}[<+->]
\item En \cpp, il n'existe pas de « try with ressources » comme en \java
	\begin{itemize}
	\item ... et on n'en a pas besoin, grâce aux destructeurs
	\end{itemize}
\end{itemize}
\end{frame}

\begin{frame}[containsverbatim]
\frametitle{Exemple}
\begin{itemize}
\item Fichier \texttt{bad-ressource.cpp}
\end{itemize}
\begin{lstlisting}
void f()
{
	r1.acquire();
	r2.acquire();

	throw 1;
		
	r2.liberate();
	r1.liberate();	
}
\end{lstlisting}
\end{frame}

\begin{frame}[containsverbatim]
\frametitle{Exemple}
\begin{itemize}
\item Fichier \texttt{mediocre-ressource.cpp}
\end{itemize}
\begin{lstlisting}
void f()
{
	r1.acquire();
	r2.acquire();
	
	try
	{
		throw 1;
	}
	catch(...)
	{
		r2.liberate();
		r1.liberate();

		throw; // rethrow
	}
		
	r2.liberate();
	r1.liberate();	
}
\end{lstlisting}
\begin{itemize}
\item Propice aux erreurs
	\begin{itemize}
	\item Copier / coller
	\item Les programmeurs s'ennuient
	\end{itemize}
\end{itemize}
\end{frame}

\begin{frame}[containsverbatim]
\frametitle{Exemple}
\begin{itemize}
\item Fichier \texttt{good-ressource.cpp}
\end{itemize}
\begin{lstlisting}
class ressource_ptr {
	private:
		ressource* ptr;
		bool acquired;

	public:
		ressource_ptr(ressource& r) : ptr(&r), acquired(r.acquire()) {}

		virtual ~ressource_ptr() {
			acquired = ptr->liberate();
		}
		
		bool operator() ()  {  return acquired; }
};

void f() {
	ressource_ptr ptr1(r1);
	ressource_ptr ptr2(r2);

	throw 1;
}
\end{lstlisting}
\begin{itemize}
\item Plus de détails (RAII) dans le chapitre 10, grâce à la sémantique de déplacement
\end{itemize}
\end{frame}

\section{Choix de gestionnaire}

\begin{frame}
\frametitle{Règle de base}
\begin{alertblock}<+->{Règle}
	\begin{itemize}[<+->]
	\item L'ordre dans lequel sont spécifiées les \texttt{catch} a de l'importance.
	\end{itemize}
\end{alertblock}
\begin{itemize}[<+->]
\item Similaire à ce qui se passe en \java.
\item Quand une exception est lancée, le gestionnaire d'exceptions recherche un bloc \texttt{catch} « approprié » suivant certaines propriétés.
\item Si aucun \texttt{catch} approprié n'est trouvé, le gestionnaire renvoie l'exception à la fonction appelante.
	\begin{itemize}
	\item On cherche un bloc \texttt{catch} approprié chez l'appelant.
	\end{itemize}
\item On répète ce processus jusqu'à la fonction \texttt{main}. Si aucun bloc \texttt{catch} correct n'est trouvé, la fonction \texttt{terminate} est appelée.
\end{itemize}
\end{frame}

\begin{frame}
\frametitle{Priorité de recherche}
\begin{itemize}[<+->]
\item Quand une exception de type \texttt{T} est lancée, on cherche avant tout, de haut en bas (séquentiellement)
	\begin{enumerate}
	\item une correspondance exacte
		\begin{itemize}
		\item \lstinline|T|, \lstinline|T&|, \lstinline|const T|, \lstinline|const T&|
		\item \lstinline|const| n'intervient pas : transmission par valeur
		\end{itemize}
	\item une classe de base \texttt{S} de \texttt{T}
		\begin{itemize}
		\item Bonne pratique : spécifier les \lstinline|catch| de classes dérivées avant les \lstinline|catch| des classes de base.
		\end{itemize}	
	\item un gestionnaire de type quelconque
		\begin{itemize}
		\item \lstinline|catch(...)|
		\end{itemize}
	\end{enumerate}
\item Dès qu'un gestionnaire correspond, on l'exécute sans se préoccuper des autres.
\item Aucune conversion implicite n'est effectuée, même non dégradante.
\end{itemize}
\end{frame}

\begin{frame}[containsverbatim]
\frametitle{Exemple}
\begin{itemize}
\item Fichier \texttt{gest.cpp}
\end{itemize}
\begin{lstlisting}
struct exceptA {};                                                      
struct exceptB : exceptA {};                                            
struct exceptC : exceptB {};                                            

void f()
{ throw exceptB(); //throw 1.; }

int main()
{
	try             
	{
		f(); cout << "Fine" << endl;
	}
	catch(exceptA e)
		{ cout << "I caught an A" << endl; }
	catch(exceptB e)
		{ cout << "I caught a B" << endl; }
	catch(exceptC e)
		{ cout << "I caught a C" << endl; }
	catch(int d)
		{ cout << "I caught an int" << endl; }
	catch(...)
		{ cout << "I caught something" << endl; }
}
\end{lstlisting}
\end{frame}

\begin{frame}
\frametitle{Fonctions de terminaison}
\begin{itemize}[<+->]
\item Le standard fournit plusieurs fonctions permettant de terminer brutalement l'exécution d'un programme.
\item Permet de gérer un comportement inattendu du programme (bug), voire de notifier l'OS du type d'erreur rencontrée.
\item Il en existe trois :
	\begin{enumerate}
	\item \texttt{abort}
	\item \texttt{exit}
	\item \texttt{terminate} (\cpp 11)
	\end{enumerate}
\item Gestion « plus fine » que \texttt{System.exit(int)} en \java.
\item Ne pas abuser : gérer vos exceptions est de première importance.
\end{itemize}
\end{frame}

\begin{frame}
\frametitle{\texttt{abort}}
\begin{itemize}[<+->]
\item Dénote une fin « anormale » du programme
\item Change le flag POSIX \texttt{SIGABRT}
	\begin{itemize}
	\item Si un handler a été mis en place pour ce flag, il est utilisé.
	\item Le programme se termine
	\end{itemize}
\item Utilisé classiquement quand une erreur non attendue se produit, comme un bug de programmation.
\item Exemple
	\begin{itemize}
	\item Une exception qui n'est pas supposée se lancer se lance.
	\item Un pointeur est \texttt{null} alors qu'il ne devrait pas l'être.
	\end{itemize}
\end{itemize}
\end{frame}

\begin{frame}
\frametitle{\texttt{exit}}
\begin{itemize}[<+->]
\item Dénote une fin « normale » du programme.
\item Peut néanmoins indiquer un échec, mais pas un bug.
	\begin{itemize}
	\item Une expression ne peut pas être décomposée.
	\item Un fichier ne peut pas être lu.
	\end{itemize}
\item On peut invoquer \texttt{exit} avec un code d'erreur.
	\begin{itemize}
	\item Par défaut, $0$ indique une sortie du programme avec succès.
	\end{itemize}
\item Les sorties de programme avec \texttt{exit} peut également être gérées par le système d'exploitation.
	\begin{itemize}
	\item Fonctions \texttt{atexit} et \texttt{on\_exit}.
	\end{itemize}
\end{itemize}
\end{frame}

\begin{frame}
\frametitle{\texttt{std::terminate}}
\begin{itemize}[<+->]
\item Automatiquement appelé par \cpp\ quand une exception non gérée est lancée.
\item Par défaut, appelle \texttt{abort}
\item Redéfinition possible via \texttt{std::set\_terminate}
\item Habituellement, on veut gérer toutes les exceptions possibles.
	\begin{itemize}
	\item Et donc ne pas appeler \texttt{exit} ou \texttt{abort}.
	\end{itemize}
\item Utilisation de \texttt{terminate} à ne pas abuser
	\begin{itemize}
	\item Le système d'exploitation ne peut pas savoir ce qui a provoqué \texttt{terminate}.
	\item ... contrairement à \texttt{exit} et \texttt{abort}.
	\end{itemize}
\end{itemize}
\end{frame}

\section{Les exceptions standard}

\begin{frame}
\frametitle{Exceptions standard (1/2)}
\begin{itemize}
\onslide<1-> \item Il existe plusieurs classes d'exceptions standard définies dans \texttt{stdexcept.h}
\onslide<2-> 
\item \texttt{exception}
	\begin{itemize}
	\item \texttt{logic\_error}
		\begin{itemize}
		\item \texttt{domain\_error}
		\item \texttt{invalid\_argument}
		\item \texttt{length\_error}
		\item \texttt{out\_of\_range}
		\end{itemize}
	\item \texttt{runtime\_error}
		\begin{itemize}
		\item \texttt{range\_error}
		\item \texttt{overflow\_error}
		\item \texttt{underflow\_error}
		\end{itemize}
	\end{itemize}
\item \texttt{bad\_alloc}
\item \texttt{bad\_cast}
\item \texttt{bad\_exception}
\item \texttt{bad\_typeid}
\end{itemize}
\end{frame}

\begin{frame}
\frametitle{Exceptions standard (2/2)}
\begin{itemize}[<+->]
\item \texttt{bad\_alloc} : échec d'allocation mémoire par \texttt{new}
\item \texttt{bad\_cast} : échec de l'opérateur \lstinline|dynamic_cast|
\item \texttt{bad\_typeid} : échec de la fonction \texttt{typeid}
\item \texttt{bad\_exception} : erreur de spécification d'exception, parfois lancée dans certaines implémentations de \texttt{unexpected}
\item \texttt{out\_of\_range} : dépassement de bornes
\item \texttt{invalid\_argument} : paramètre d'appel invalide
\item \texttt{overflow\_error}, \texttt{underflow\_error} : lancée lors d'erreurs de calculs flottants.
\item La fonction \texttt{what} retourne un \lstinline|const char *| décrivant la nature d'une exception lancée.
\item Toutes les classes possèdent un constructeur à un argument chaîne de caractère permettant de spécifier cette chaîne.
%\item Possibilité de définir un gestionnaire particulier pour \texttt{bad\_alloc} avec \texttt{set\_new\_handler} de \texttt{new.h}.
\end{itemize}
\end{frame}

\end{document}
