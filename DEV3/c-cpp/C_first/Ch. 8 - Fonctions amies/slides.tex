\documentclass{beamer}

\usepackage[francais]{babel}
\usepackage[utf8]{inputenc}
\usepackage[T1]{fontenc}
\usepackage{graphicx}
\usepackage{graphics}
\usepackage{color}
\usepackage{textcomp}
\usepackage{pifont}
\usepackage[normalem]{ulem}
\usepackage{times}
\usepackage{hyperref}
\usepackage{verbatim}
\usepackage{amsmath}
\usepackage{amsthm}
\usepackage{amsfonts}
\usepackage[mathscr]{euscript}
\usepackage{pgfpages}
\usepackage{listings}
\usepackage{subfigure}
\usepackage{algorithm}
\usepackage[noend]{algorithmic}
\usepackage{pdftricks}
\usepackage{mathrsfs}
\usepackage{array}
\usepackage{fancybox}
% \usepackage{columns}
\usepackage{multirow}
\usepackage{url}
\usepackage{tikz}
\usepackage{colortbl}
%\usepackage{cite} %DO NOT FUCKING USE CITE ON BEAMER !!! LOST 30 GODDAM' MINUTES ON THIS SHIT !!!
\usepackage{mathabx}
\usepackage{amssymb}
\usepackage{eurosym}
\usepackage{wasysym} % ch0

\let\texteuro\euro

\hypersetup{colorlinks,%
            citecolor=black,%
            filecolor=black,%
            linkcolor=black,%
            urlcolor=blue}

%\addtolength{\parskip}{10pt}

\usetikzlibrary{calc}

\mode<presentation>
\setbeamertemplate{footline}[frame number]
\setbeamercovered{transparent}
\usetheme[navigation]{ESI}

%lst
\definecolor{comment-green}{RGB}{0,166,80}
\lstset{language=C++,
  keywordstyle=\lst@ifdisplaystyle\bf\fi\color{blue!60},
  commentstyle=\color{comment-green},
  stringstyle=\color{red},
  basicstyle=\lst@ifdisplaystyle\tiny\else\tt\fi,
  morekeywords={
    constexpr,concept,decltype,nullptr,nullptr_t,noexcept,final,override},
  frame=single,
  xleftmargin=0.5cm,
  numbers=left,
  tabsize=2}

%title
\subtitle{Langage \texttt{C} / \cpp}
\author{R. Absil}
\date{\today}

%styles
\theoremstyle{definition}
\newtheorem{thm}{Théorème}
\newtheorem{conj}[thm]{Conjecture}
\newtheorem{deff}[thm]{Définition}
\newtheorem{prop}[thm]{Propriété}
\newtheorem{lem}[thm]{Lemme}
\newtheorem*{lem*}{Lemme}
\newtheorem{cor}[thm]{Corollaire}
%\newtheorem{example}{Exemple}
\newtheorem{remark}{Remarque}
\newtheorem{exo}{Exercice}

%typeset
\newcommand{\ie}{{\emph{i.e., }}}
\newcommand{\eg}{{\emph{e.g., }}}
\newcommand{\etal}{{\emph{et al.}}}
\newcommand{\rrceil}{\unichar{"2308}}
\newcommand{\sloand}[2]{\footnote{N. J. A. Sloane - OEIS Foundation - \texttt{www.oeis.org}, Sequence #1 - #2.}}

%math
\newcommand{\IN}{{\mathbb N}}
\newcommand{\IQ}{{\mathbb Q}}
\newcommand{\IR}{{\mathbb R}}
\newcommand{\IZ}{{\mathbb Z}}
\newcommand{\IP}{{\mathbb P}}
\newcommand{\IC}{{\mathbb C}}
\newcommand{\bigo}{{\mathcal{O}}}
\renewcommand{\mod}{\bmod}
\newcommand{\ssi}{\Leftrightarrow}
\newcommand{\then}{\Rightarrow}
\newcommand{\fle}[1]{\stackrel{#1}{\longrightarrow}}
\newcommand{\suchthat}{~\big|~}
\newcommand{\floor}[1]{\left\lfloor #1 \right\rfloor}
\newcommand{\ceil}[1]{\left\lceil #1 \right\rceil}
\DeclareMathOperator*{\argmin}{argmin}
\DeclareMathOperator*{\argmax}{argmax}

%tikz
\tikzstyle{_vertex}=[fill=white, circle,minimum size=12pt,inner sep=1pt]
\tikzstyle{_blackv}=[fill=black, circle,minimum size=8pt,inner sep=1pt]
\tikzstyle{_dot}=[fill=black, circle, minimum size = 1mm, inner sep=0pt]
\tikzstyle{_bigvertex}=[fill=white, circle,minimum size=21pt,inner sep=1pt]
\tikzstyle{_arc}=[->, >=stealth]
\tikzstyle{_boldarc}=[->, >=stealth, line width=2pt]

\newcommand{\cpp}{\texttt{C++}}
\newcommand{\java}{\texttt{Java}}


\title{Ch. 7 - Fonctions amies}

\begin{document}
\begin{frame}
  \titlepage
\end{frame}

\begin{frame}
  \frametitle{Table des matières}
  \footnotesize \tableofcontents[pausesections,pausesubsections]
\end{frame}


\section{Introduction}

\begin{frame}
\frametitle{Contexte}
\begin{itemize}[<+->]
\item La POO « impose » l'encapsulation des données
	\begin{itemize}
	\item Les membres privés ne sont accessibles qu'à l'intérieur de la classe
	\item Usage de getters et setters
	\end{itemize}
\end{itemize}
\begin{alertblock}<+->{Inconvénient}
	\begin{itemize}[<+->]
	\item On peut être amené à faire beaucoup de \texttt{call}
	\item Légère perte de performance
	\end{itemize}
\end{alertblock}
\begin{exampleblock}<+->{Solutions connues}
	\begin{enumerate}[<+->]
	\item Changer les membres en \lstinline|public|
	\item Utiliser des accesseurs / mutateurs \lstinline|inline|
	\end{enumerate}
\end{exampleblock}
\end{frame}

\begin{frame}[containsverbatim]
\frametitle{Exemple de classe publique}
\begin{itemize}
\item Fichier \texttt{public.cpp}
\end{itemize}
\begin{lstlisting}
class point
{	
	public :
		double x,y;

		point(double abs=0, double ord=0) : x(abs), y(ord) {}

		double distance(point p)
		{
			return sqrt((x - p.x) * (x - p.x) 
				+ (y - p.y) * (y - p.y));
		}

		string toString()
		{
			stringstream str;
			str << "( " << x << " , " << y << " )";		
			return str.str();
		}
};
\end{lstlisting}
\end{frame}

\begin{frame}[containsverbatim]
\frametitle{Exemple de classe publique}
\begin{itemize}
\item Fichier \texttt{public.cpp}
\end{itemize}
\begin{lstlisting}
class circle
{
	point center;
	double radius;

	public :
		circle(point p, double rad) : center(p), radius(rad) {}

		inline void translate(double x, double y)
		{
			center.x += x;
			center.y += y;
		}

		string toString()
		{			
			stringstream str;
			str << "Circle of center " << center.toString() << " and of radius " << radius;
			return str.str();
		}
};
\end{lstlisting}
\end{frame}

\begin{frame}[containsverbatim]
\frametitle{Exemple de classe publique}
\begin{itemize}
\item Fichier \texttt{public.cpp}
\end{itemize}
\begin{lstlisting}
int main()
{
	point p (1,1);
	circle c (p, 2);

	cout << p.toString() << endl;
	cout << c.toString() << endl;

	c.translate(1,-1);

	cout << p.toString() << endl;
	cout << c.toString() << endl;	
}
\end{lstlisting}
\end{frame}

\begin{frame}
\frametitle{Debriefing}
\begin{alertblock}<+->{Problèmes}
	\begin{itemize}[<+->]
	\item Violation d'encapsulation
	\item « Toute le monde » a accès en lecture et écriture sur les attributs
	\end{itemize}
\end{alertblock}
\begin{block}<+->{Avantages}
	\begin{itemize}[<+->]
	\item Gain de performances
		\begin{itemize}
		\item Pas de \texttt{call}
		\end{itemize}
	\item Écriture concise
	\end{itemize}
\end{block}
\end{frame}

\begin{frame}[containsverbatim]
\frametitle{Exemple de classe avec accesseurs / mutateurs}
\begin{itemize}
\item Fichier \texttt{access.cpp}
\end{itemize}
\begin{lstlisting}
class point
{	
	double _x, _y;//ask why I wrote _x

	public :		
		point(double abs=0, double ord=0) : _x(abs), _y(ord) {}

		double distance(point p)
		{
			return sqrt((_x - p._x) * (_x - p._x) 
				+ (_y - p._y) * (_y - p._y));
		}

		string toString()
		{
			stringstream str;
			str << "( " << _x << " , " << _y << " )";		
			return str.str();
		}

		inline void set(int abs, int ord) { _x = abs; _y = ord; }
		inline double x() const { return _x; }
		inline double y() const { return _y; }
};
\end{lstlisting}
\end{frame}

\begin{frame}[containsverbatim]
\frametitle{Exemple de classe avec accesseurs / mutateurs}
\begin{itemize}
\item Fichier \texttt{access.cpp}
\end{itemize}
\begin{lstlisting}
class circle
{
	point _center;
	double _rad;

	public :
		circle(point p, double rad) : _center(p), _rad(rad) {}		

		void translate(double x, double y)
		{
			_center.set(_center.x() + x, _center.y() + y);
		}

		string toString()
		{			
			stringstream str;
			str << "Circle of center " << _center.toString() << " and of radius " << _rad;
			return str.str();
		}
};
\end{lstlisting}
\end{frame}

\begin{frame}
\frametitle{Debriefing}
\begin{alertblock}<+->{Problèmes}
	\begin{itemize}[<+->]
	\item Exécutable potentiellement plus gros
	\end{itemize}
\end{alertblock}
\begin{block}<+->{Avantages}
	\begin{itemize}[<+->]
	\item Encapsulation respectée
	\item Obligation de passer par des accesseurs / mutateurs
	\item Écriture assez concise
	\end{itemize}
\end{block}
\end{frame}

\section{Fonctions amies}

\begin{frame}
\frametitle{Overview}
\begin{itemize}[<+->]
\item Les solutions connues ont des inconvénients qui ne peuvent toujours être ignorés
\end{itemize}
\begin{exampleblock}<+->{Idée}
	\begin{itemize}
	\item Possibilité de déclarer une fonction / classe \emph{amie} à la définition d'une classe
	\item Autorise l'accès aux attributs privés pour cette fonction / classe
	\end{itemize}
\end{exampleblock}
\begin{itemize}[<+->]
\item Plusieurs situations d'amitié possibles
	\begin{enumerate}
	\item Fonction indépendante amie d'une classe
	\item Fonction membre d'une classe A amie d'une classe B
	\item Toutes les fonctions d'une classe sont amies d'une autre classe
	\end{enumerate}
\item Utilisation du mot-clé \lstinline|friend|
\end{itemize}
\end{frame}

\begin{frame}
\frametitle{Avantage}
\begin{itemize}[<+->]
\item Le \emph{concepteur} d'une classe spécifie quelles autres fonctions / classes sont ses amis
\item Les amis ont accès aux membres privés comme s'ils étaient publics
\item Très efficace
	\begin{itemize}
	\item Pas de \texttt{call}
	\end{itemize}
\item Compromis d'encapsulation
	\begin{itemize}
	\item Les attributs ne sont pas visibles depuis l'extérieur
	\item ... sauf si le concepteur l'a explicitement autorisé
	\end{itemize}
\end{itemize}
\begin{exampleblock}<+->{Remarque}
	\begin{itemize}[<+->]
	\item \lstinline|friend| peut déclarer un prototype
	\item Attention à l'ordre et aux inclusions de fichier
	\end{itemize}
\end{exampleblock}
\end{frame}

\section{Relations d'amitié}

\begin{frame}
\frametitle{Remarques techniques}
\begin{itemize}[<+->]
\item Une déclaration d'amitié permet d'accéder aux membres privés (et protégés)
\item Offre de meilleurs performances
\item Une déclaration d'amitié peut déclarer un prototype de fonction
	\begin{itemize}
	\item Mais pas un prototype de classe
	\end{itemize}
\item \emph{Souvent}, l'endroit où une déclaration d'amitié au sein d'une classe est déclarée n'a pas d'importance
	\begin{itemize}
	\item Avec des templates (cf. Ch. 13)
	\end{itemize}
\end{itemize}
\begin{block}<+->{Hygiène de programmation}
	\begin{itemize}[<+->]
	\item Ne pas abuser
	\end{itemize}
\end{block}
\end{frame}

\subsection{Fonction indépendante amie d'une classe}

\begin{frame}[containsverbatim]
\frametitle{Exemple}
\begin{itemize}
\item Fichier \texttt{indep.cpp}
\end{itemize}
\begin{lstlisting}
class point
{
	int x, y;
	
	public:
		point(int abs = 0, int ord = 0)  : x(abs), y(ord) {}
		
		friend bool coincide(const point &, const point &);
};

int main()
{
	point a(1,0), b(1), c;
	if(coincide(a,b))
		cout << "a coincide avec b" << endl;
	else
		cout << "a et b sont différents" << endl;		
	if(coincide(a,c))
		cout << "a coincide avec c" << endl;
	else
		cout << "a et c sont différents" << endl;
}

bool coincide(const point & p, const point & q) { return ((p.x == q.x) && (p.y == q.y)); }
\end{lstlisting}
\end{frame}

\begin{frame}
\frametitle{Remarques}
\begin{itemize}[<+->]
\item L'emplacement de la déclaration d'amitié de \texttt{coincide} au sein de \texttt{point} n'a pas d'importance
\item Cette déclaration déclare également un prototype
	\begin{itemize}
	\item Possibilité d'utiliser \texttt{coincide} dans \texttt{main} avant une « vraie » déclaration
	\end{itemize}
\item La déclaration d'amitié déclare une fonction indépendante car absence de paramètre implicite
	\begin{itemize}
	\item Deux paramètres, pas de classe, etc.
	\end{itemize}
\item En général, une fonction amie d'une classe possède un ou plusieurs arguments du type de cette classe
	\begin{itemize}
	\item Ce n'est pas une obligation
	\end{itemize}
\item Possibilité de viol d'encapsulation
	\begin{enumerate}
	\item Savoir qu'une fonction indépendante est amie d'une classe
	\item Obtenir le prototype de la fonction indépendante
	\item Réécrire une fonction de même prototype
	\end{enumerate}
\end{itemize}
\end{frame}

\subsection{Fonction membre amie d'une classe}

\begin{frame}[containsverbatim]
\frametitle{Exemple}
\begin{itemize}
\item Fichier \texttt{member.cpp}
\end{itemize}
\begin{lstlisting}
class A;  // forward declaration of A needed by B

class B
{
	public:
	    void fB(A& a);
};

class A
{
	int i;
	public:
		//specifying function fA as a friend of A, fA is not member function of A
		friend void fA(A& a);    
	    
		//specifying B class member function fB as a friend of A
		friend void B::fB(A& a); 
};
\end{lstlisting}
\end{frame}

\begin{frame}[containsverbatim]
\frametitle{Exemple}
\begin{itemize}
\item Fichier \texttt{member.cpp}
\end{itemize}
\begin{lstlisting}
// fA is Friend function of A
void fA(A& a)
{
    a.i  = 11; // accessing and modifying Class A private member i
    cout << a.i << endl;
}

// B::fB should be defined after class A definition
void B::fB(A& a)
{
    a.i  = 22; 
    cout << a.i << endl;
}

int main()
{
    A a;
    fA(a);    // calling friend function of class A

    B b;
    b.fB(a);  // calling B class member function fB, B:fB is friend of class A
}
\end{lstlisting}
\end{frame}

\subsection{Classe amie d'une classe}

\begin{frame}[containsverbatim]
\frametitle{Exemple}
\begin{itemize}
\item Fichier \texttt{class.cpp}
\end{itemize}
\begin{lstlisting}
class B;

class A
{
	int _i;
	
	public:
		A(int i) : _i(i) {}
		int i() const { return _i; }

		friend class B;
};

class B
{
	A a;
	int _j;
	
	public:
		B(A a, int j) : a(a), _j(j) {}

		int brol() const { return a._i * _j; }	
};
\end{lstlisting}
\end{frame}

\begin{frame}
\frametitle{Propagation d'une relation d'amitié}
\begin{alertblock}<+->{Règles}
\begin{itemize}[<+->]
\item Les relations d'amitié ne sont pas transitives
	\begin{itemize}
	\item L'ami d'un ami n'est pas votre ami
	\end{itemize}
\item L'amitié n'est pas propagée par héritage (Cf. Ch. 11)
	\begin{itemize}
	\item Les enfants de votre ami ne sont pas vos amis	
	\item Vos enfants ne sont pas les amis de votre ami
	\end{itemize}
\item À partir de \cpp11, les amis ont accès aux classes internes privées
\end{itemize}
\end{alertblock}
\begin{itemize}[<+->]
\item Souvent, un choix de design est effectué pour soit
	\begin{enumerate}
	\item rendre une classe \texttt{B} entière amie d'une autre classe \texttt{A}
	\item faire d'une classe \texttt{B} une classe interne d'une autre classe \texttt{A}
	\end{enumerate}
\item Règles particulières dans le cas de templates (Cf. Ch. 13)
\end{itemize}
\end{frame}

\begin{frame}[containsverbatim]
\frametitle{Exemple}
\begin{itemize}
\item Fichier \texttt{subclassing.cpp}
\end{itemize}
\begin{lstlisting}
class A
{
	int _i;
	
	public:
		A() : _i(2) {}
		int i() const { return _i; }

		friend class M; //class M is a friend of A
};

class B : public A //A is a subclass of A
{
	int _j;
	
	public:
		B() : _j(3) {}
		int j() const { return _j; }
};
\end{lstlisting}
\end{frame}

\begin{frame}[containsverbatim]
\frametitle{Exemple}
\begin{itemize}
\item Fichier \texttt{subclassing.cpp}
\end{itemize}
\begin{lstlisting}
//M is a friend of A and not a friend of its children
class M
{
	int _k;

	public:
		M(A a) : _k(a._i * 2) {}
		//M(B b) : _k(b._j * 3) {}

		int k() const { return _k; }
};

//children of M are neither friends of A or B
class N : public M
{
	int _l;

	public:
		N(A a) : M(a)/*, _l(a._i * 4)*/ {}
		N(B b) : M(b)/*, _l(b._j * 5)*/ {}

		int l() const { return _l; }
};
\end{lstlisting}
\end{frame}
\end{document}
